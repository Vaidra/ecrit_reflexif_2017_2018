\documentclass[a4paper, 12pt]{article}

\usepackage{helvet}
\renewcommand{\familydefault}{\sfdefault}

%\hyphenation
\usepackage{pdfpages}

\usepackage[utf8]{inputenc}  
\usepackage[francais]{babel}
\usepackage[T1]{fontenc}  
\renewcommand{\baselinestretch}{1.5} 

\title{Écrit réflexif : L'évaluation par compétences}
\author{CLIN Exavérine}
\date{2017-2018}

\begin{document}
\maketitle
\thispagestyle{empty}

\newpage
\part*{Remerciements}
% Remerciements

Je remercie Directrice ESPE


IAIPR


Monsieur Jolly, chef d'établissement de m'avoir accueillit


Je tiens à remercier mon tuteurs EPLE, Patrick PARENT, pour son soutien tout au long de cette première année d'exercice, ses précieux conseils ainsi que sa disponibilité.

Je remercie également l'ensemble des formateurs ESPE pour les formations qu'ils ont dispensés.




\thispagestyle{empty}
%\addcontentsline{toc}{part}{Remerciements} 

\newpage
\part*{Résumé}
\addcontentsline{toc}{part}{Résumé} 
\input{parties/resume}

%\addcontentsline{toc}{part}{Le Monde} 
%\addcontentsline{toc}{chapter}{L'Eurasie} 
%\addcontentsline{toc}{section}{L'Europe} 
%\addcontentsline{toc}{subsection}{La France} 
%\addcontentsline{toc}{subsubsection}{L'Aquitaine} 
%\addcontentsline{toc}{paragraph}{La Gironde} 
%\addcontentsline{toc}{subparagraph}{Bordeaux} 

\newpage
\tableofcontents

\newpage
\part{Introduction}

% Présentation du collège
J'enseigne dans le collège Henry de Montherlant situé à Neuilly-en-Thelle.
Ce collège est un collège sans note, je me suis demandée comment évaluer les élèves moi qui étais habitué aux notes.

% Situation déclanchante
Lors de l'évaluation sommative sur la première séquence j'ai constaté une faible réussite.
J'ai donc remis en cause tout d'abord mon enseignement puis me suis ensuite interrogée sur la façon d'évaluer les élèves.

% Problématique
J'en suis donc arrivé à la problématique suivante : 
Comment évaluer les élèves ?
Quels sont les outils qui permettent une évaluation par compétence?
Comment peut-on rendre les élèves acteurs de leur évaluation ?



\newpage
\part{Situation déclenchante}
% Mise en place de l'évaluation

\section{Évaluation des connaissances}

Ma première séquence était intitulée \og Qu'est-ce qu'un cahier des charges fonctionnelles~?\fg. 
Pour cette première séquence je n'ai pas réalisé d'évaluation diagnostique afin de déterminer les connaissances des élèves sur le cahier des charges.
Cela fut une première erreur de ma part car je pensais effectuer des rappels avant d'aller plus loin, or pour les élèves les notions de \og diagramme bête à cornes\fg et de \og diagramme pieuvre\fg n'évoquait rien. J'ai donc passé une séance complète sur la réalisation de ces deux diagrammes qui étaient inconnus aux élèves.
Si j'avais effectué une rapide évaluation diagnostique cela m'aurait fait gagner du temps.

Concernant la structure de la séquence que j'ai réalisé sur la réalisation du cahier des charges je demandais énormément d'attention de la part des élèves.
Effectivement mon cours s'est déroulé de façon magistral avec une partie travaux dirigés durant laquelle les élèves ont réalisés les deux types de diagrammes.
Les élèves n'ont pas communiqués durant les exercices et s'en ai suivie une phase de correction au tableau par des élèves volontaire.

Afin de vérifier que les élèves puissent faire par eux même un diagramme bête à cornes je leur ai demandé de réaliser sur une feuille ce diagramme sur différents objets de leur quotidien.

J'ai ensuite poursuivit le cours sur le cahier des charges notamment sur les fonctions principales et fonctions contraintes.

A la fin de la séquence j'ai prévenu les élèves qu'une évaluation aurait lieu la semaine suivante. Cette évaluation porterait sur la réalisation de deux diagrammes et l'identification de la fonction principale et des fonctions contraintes.

L'évaluation que j'ai réalisé se trouve Annexe~\ref{annexe:evaluation_cdcf}.

\section{Critique}

Cette évaluation est une évaluation dite \og classique \fg qui consiste à évaluer les connaissances des élèves.


\setcounter{section}{0}

\newpage
\part{Définitions de l'évaluation}
\section{L'évaluation}

Au sens étymologique le terme "évaluation" provient de "ex-valuere" qui signifie déterminer la valeur de.

Dans la bibliographie il existe de nombreuses définitions de l'évaluation, mais celle de \cite{de_ketele_levaluation_1989} reste parmi les plus complètes.
\og Évaluer signifie
\begin{itemize} 
\item recueillir un ensemble d'informations suffisamment pertinentes, valides et fiables
\item et examiner le degré d'adéquation entre cet ensemble d'informations et un ensemble de critères adéquats aux objectifs fixés au départ ou ajustés en cours de route
\item en vue de prendre une décision \fg
\end{itemize}


Il explique que l'évaluation prépare à une décision mais n'en fait pas partie. 
Le fait de parler de décision est assimilé à un jugement, au fait d'apprécier une personne ou une action. 
Mais l'évaluation, contrairement au jugement, est un processus basé sur des critères explicites.

Pour cela dans son livre, ~\cite{roegiers_lecole_2010} propose trois évaluations des acquis des élèves~:
\begin{itemize}
\item \og en début d'année sur les performances des élèves pour décider si l'on peut commencer les apprentissages comme prévu \fg ~;
\item \og en cours d'année sur les performances d'un élève pour décider s'il est nécessaire de mettre en place une procédure de remédiation \fg ~;
\item \og en fin d'année sur les performances d'un élève pour décider s'il peut passer dans la classe supérieure \fg .
\end{itemize}

Ces trois types d'évaluations se déroule à deux échelles celle de l'élève et celle de la classe.
Concernant l'élève~:
\begin{itemize}
\item L'évaluation qui s'effectue avant l'apprentissage s'appelle l'évaluation d'orientation. Elle permet de diagnostiquer les forces et les faiblesses pour orienter l'élève vers un type d'apprentissage plus adapté. C'est pour cela que l'on parle également d'évaluation diagnostique.
\item L'évaluation qui s'effectue durant l'apprentissage permet de détecter les difficultés au niveau de chaque élève afin d'y remédier. Cette évaluation s'appelle évaluation formative.
\item L'évaluation qui s'effectue en fin d'apprentissage, fin d'année, permet de valider les acquis des élèves, c'est l'évaluation sommative aussi appelée évaluation certificative lorsqu'elle aboutit à la délivrance d'un diplôme ou l'accès à la formation du niveau supérieur.
\end{itemize}


%%% ----------------- Dans le collège

\section{Mise en place au sein de mon établissement}

Il y a quatre niveaux d'acquisition sur les compétences~:
\begin{itemize}
\item Niveau 1~: Non acquis, ce niveau est identifié dans le collège par deux points rouges.
\item Niveau 2~: En voie d'acquisition, identifié par un point rouge.
\item Niveau 3~: Acquis, identifié par un point vert. L'élève à atteint l'objectif
\item Niveau 4~: Expert, identifié par deux points verts L'élève à largement atteint l'objectif qui était fixé.
\end{itemize}

Le suivi des compétences des élèves se fait grâce à l'outil Sacoche qui permet de saisir lors d'une évaluation les compétences ainsi que le niveau atteint par chacun des élèves.

L'évaluation par compétence permet de mettre en évidence les différents savoirs à la différence d'une note globale qui n'identifie pas les points forts et difficultés de l'élève.

%%% -----------------
%%% -----------------

\section{Les compétences}

On identifie quelqu'un de compétent comme ayant des acquis (connaissances, savoir-faire, procédures, etc.) et sachant les mobiliser pour résoudre un problème donné.

Les savoirs regroupe la connaissance d'un vocabulaire.
Le savoir-faire
Le savoir-être


\setcounter{section}{0}

\newpage
\part{Mise en place d'une évaluation}

%%% ------------------------

\section{Évaluation durant les travaux pratiques}

Durant la séquence 2 intitulée \og Comment aménager un lotissement~? \fg \ j'ai commencé par présenter les compétences travaillées par les élèves durant la première séance.
Puis j'ai expliqué l'activité sur laquelle ils seront évalués~: afin de faire venir de nouveaux habitants dans sa commune, le maire de Neuilly-en-Thelle souhaite créer un nouveau lotissement pour accueillir des terrains à bâtir.
L'objectif est d'installer un maximum de terrains à bâtir dans l'espace disponible mais afin de garantir le confort des nouveaux habitants, le constructeur préconise des dimensions à respecter.
Bien entendu chaque terrain à bâtir doit être relié à la route.

Pour cela les élèves sont répartis par groupe de 4 et ont à leur disposition un plan du lotissement à l'échelle $1:50$, des rectangles, chacun représentant un terrain à bâtir avec les dimensions imposées ainsi que des feuilles dans lesquelles ils devront découper les routes, la largeur des routes est aussi imposée.

Les élèves travaillent sur le cahier des charges ainsi que la représentation d'une solution technique à un problème posé.
Ils sont évalués sur leur réalisation, c'est-à-dire le respect du cahier des charges ainsi que l'optimisation de l'espace occupé.

Durant l'activité les élèves étaient investis dans leur travail, ils ont testé plusieurs répartitions spatiales afin de trouver la plus optimisée.
Les compétences que les élèves ont travaillées sont majoritairement acquises excepté un groupe qui n'a positionné aucune route pour relier les terrains à bâtir au reste de la commune.


%%% ------------------------

\section{Donner les clés de la réussite aux élèves}

Lors du lancement de la séquence 4 intitulée \og Comment évoluent les objets techniques~?\fg j'ai présenté les compétences travaillées mais j'ai aussi fourni aux élèves une fiche qui détaille les objectifs à atteindre pour les différents niveaux d'acquisition. 
Chaque compétence est découpée en 4 objectifs décrit sur la feuille distribuée, chaque objectif validé rapporte 1 point. Suivant le nombre de point sur chaque compétence l'élève obtient soit Insuffisant (0-1 point), Fragile (2 points), Acquis (3 points), Expert (4 Points).

L'objectif de l'activité évaluée était de réaliser, par binôme, une frise chronologique sur l'évolution d'un objet technique de leur choix avec le logiciel OpenOffice Dessin.

Lors de l'activité les élèves étaient impliqués dans leur évaluation, ils ont régulièrement regarder la fiche d'évaluation pour se positionner.


\section{Permettre aux élèves de choisir leur évaluation}

A la fin de la séquence 5 intitulée \og Comment fonctionnent les objets techniques~?\fg j'ai mis en place une évaluation différenciée.

Durant cette séquence les élèves ont étudiés la chaîne d'énergie de différents objets techniques ainsi que les blocs fonctionnels qui la compose.

Afin de vérifier l'acquisition de la compétence j'ai programmé une évaluation sur feuille.
Étant donné que durant les travaux dirigés certains élèves étaient en difficulté j'ai décider de mettre en place une évaluation différenciée, c'est-à-dire que j'ai proposé aux élèves deux sujets.

Le premier sujet parcours toutes les connaissances vues en classe et demande tout d'abord de représenter le schéma de la chaîne d'énergie sans aucune indication puis de réaliser la chaîne d'énergie de différents objets techniques.
Ce sujet (voir Annexe~\ref{annexe:evaluation_chaine_energie_A}) évalue la compétence sur tout le niveau d'acquisition (Insuffisant, Fragile, Acquis, Expert).

Le second (voir Annexe~\ref{annexe:evaluation_chaine_energie_B}) sujet s'adresse aux élèves ayant eu des difficulté durant les travaux dirigés.
Il évalue la compétence sur les trois premier niveaux d'acquisition, cela signifie que les élève qui ont choisi ce sujet ne pourrons pas valider la compétence avec le niveau expert.
En effet, ce sujet demande de compléter le schéma de la chaîne d'énergie qui est représenté puis de réaliser ce schéma avec plusieurs objets techniques qui ont étés vu en cours.



J'ai remarqué que certains élèves ne savent pas se positionner sur leur connaissances et compétences.




%%% ------------------------

\section{Permettre à chacun de se situer }

La séquence 7 est intitulée \og Comment fonctionne une écluse~? \fg{}.
Cette séquence se déroule en deux parties~:
\begin{itemize}
\item Découverte des éléments de l'ouvrage d'art puis du fonctionnement de celui-ci en utilisant une application flash. Les élèves devront réaliser une présentation en utilisant \textit{OpenOffice Présentation} dans laquelle ils détailleront les étapes pour permettre à l'écluse de laisser passer un bateau.
\item Étude des étapes de fonctionnement, après une explication sur la réalisation d'un organigramme, les élèves devront réaliser l'organigramme de l'écluse puis de divers objets techniques.
\end{itemize} 

Durant la présentation de la première partie sur le fonctionnement d'une l'écluse, j'ai expliqué aux élèves qu'ils seront évalués sur la réalisation d'une présentation sur le fonctionnement d'une l'écluse.
Pour cela, je leur ai présenté puis distribué la fiche d'évaluation sur laquelle je détaille les objectifs à atteindre pour valider la compétence (voir Annexe~\ref{annexe:evaluation_fct_ecluse}).
En plus de cette évaluation, je demande un retour des élèves sur leur ressentit durant l'activité.

La seconde activité porte sur la réalisation d'organigrammes.

Mise en place d'une évaluation différenciée.


\section{Associer les compétences aux questions et objectifs}

Pour la séquence suivante je penses intégrer directement l'évaluation dans le sujet que je proposerai aux élèves.
Au fur et à mesure de l'avancée des élèves sur le sujets ils auront 

%%% ------------------------



\setcounter{section}{0}

%\newpage
%\part{??}
%\input{parties/4-eleves}

\newpage
\part{Conclusion}
antibi : la constante macabre







noter les compétences sur les questions des évaluations

possibilité de donner les sujets plus compliqués aux élèves ayant fait le sujet plus faibles comme travail à la maison


\newpage
\part*{Bibliographie}
\addcontentsline{toc}{part}{Bibliographie} 
\bibliographystyle{apalike}%plain}
\bibliography{evaluation_competence}


\newpage
\part*{Annexes}
\addcontentsline{toc}{part}{Annexes} 
% Annexe 1 : CDCF
\section{Evaluation cahier des charges fonctionnel}\label{annexe:evaluation_cdcf}
\includegraphics[scale=0.7]{./ressources/Controle_CDCF.pdf} 

\section{Fiche évaluation sur la réalisation d'une frise chronologique}\label{annexe:evaluation_frise}
\includegraphics[scale=0.7]{./ressources/TP_evolution_objet_technique_4e.pdf} 

\section{Evaluation différenciée sujet A}\label{annexe:evaluation_chaine_energie_A}
\includegraphics[scale=0.7]{./ressources/Controle_chaine_energie_A.pdf} 

\section{Evaluation différenciée sujet B}\label{annexe:evaluation_chaine_energie_B}
\includegraphics[scale=0.7]{./ressources/Controle_chaine_energie_B.pdf} 

\section{Fiche évaluation sur le fonctionnement de l'écluse}\label{annexe:evaluation_fct_ecluse}
\includegraphics{./ressources/Fonctionnement_Ecluse_S1_4e_a5_1.pdf} 

%\section{Evaluation cahier des charges fonctionnel}\label{annexe:evaluation_cdcf}

% Fiche séquence evaluation

\end{document}
