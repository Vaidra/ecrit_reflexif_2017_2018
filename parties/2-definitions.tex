\section{L'évaluation}

Au sens étymologique le terme "évaluation" provient de "ex-valuere" qui signifie déterminer la valeur de.

Dans la bibliographie il existe de nombreuses définitions de l'évaluation, mais celle de \cite{de_ketele_levaluation_1989} reste parmi les plus complètes.
\og Évaluer signifie
\begin{itemize} 
\item recueillir un ensemble d'informations suffisamment pertinentes, valides et fiables
\item et examiner le degré d'adéquation entre cet ensemble d'informations et un ensemble de critères adéquats aux objectifs fixés au départ ou ajustés en cours de route
\item en vue de prendre une décision \fg
\end{itemize}


Il explique que l'évaluation prépare à une décision mais n'en fait pas partie. 
Le fait de parler de décision est assimilé à un jugement, au fait d'apprécier une personne ou une action. 
Mais l'évaluation, contrairement au jugement, est un processus basé sur des critères explicites.

Pour cela dans son livre, ~\cite{roegiers_lecole_2010} propose trois évaluations des acquis des élèves~:
\begin{itemize}
\item \og en début d'année sur les performances des élèves pour décider si l'on peut commencer les apprentissages comme prévu \fg ~;
\item \og en cours d'année sur les performances d'un élève pour décider s'il est nécessaire de mettre en place une procédure de remédiation \fg ~;
\item \og en fin d'année sur les performances d'un élève pour décider s'il peut passer dans la classe supérieure \fg .
\end{itemize}

Ces trois types d'évaluations se déroule à deux échelles celle de l'élève et celle de la classe.
Concernant l'élève~:
\begin{itemize}
\item L'évaluation qui s'effectue avant l'apprentissage s'appelle l'évaluation d'orientation. Elle permet de diagnostiquer les forces et les faiblesses pour orienter l'élève vers un type d'apprentissage plus adapté. C'est pour cela que l'on parle également d'évaluation diagnostique.
\item L'évaluation qui s'effectue durant l'apprentissage permet de détecter les difficultés au niveau de chaque élève afin d'y remédier. Cette évaluation s'appelle évaluation formative.
\item L'évaluation qui s'effectue en fin d'apprentissage, fin d'année, permet de valider les acquis des élèves, c'est l'évaluation sommative aussi appelée évaluation certificative lorsqu'elle aboutit à la délivrance d'un diplôme ou l'accès à la formation du niveau supérieur.
\end{itemize}


%%% ----------------- Dans le collège

\section{Mise en place au sein de mon établissement}

Il y a quatre niveaux d'acquisition sur les compétences~:
\begin{itemize}
\item Niveau 1~: Non acquis, ce niveau est identifié dans le collège par deux points rouges.
\item Niveau 2~: En voie d'acquisition, identifié par un point rouge.
\item Niveau 3~: Acquis, identifié par un point vert. L'élève à atteint l'objectif
\item Niveau 4~: Expert, identifié par deux points verts L'élève à largement atteint l'objectif qui était fixé.
\end{itemize}

Le suivi des compétences des élèves se fait grâce à l'outil Sacoche qui permet de saisir lors d'une évaluation les compétences ainsi que le niveau atteint par chacun des élèves.

L'évaluation par compétence permet de mettre en évidence les différents savoirs à la différence d'une note globale qui n'identifie pas les points forts et difficultés de l'élève.

%%% -----------------
%%% -----------------

\section{Les compétences}

On identifie quelqu'un de compétent comme ayant des acquis (connaissances, savoir-faire, procédures, etc.) et sachant les mobiliser pour résoudre un problème donné.

Les savoirs regroupe la connaissance d'un vocabulaire.
Le savoir-faire
Le savoir-être

