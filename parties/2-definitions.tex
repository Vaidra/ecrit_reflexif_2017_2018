\section{L'évaluation}

Au sens étymologique le terme "évaluation" provient de "ex-valuere" qui signifie déterminer la valeur de.

\subsection{Différents types d'évaluation}

Dans la bibliographie il existe de nombreuses définitions de l'évaluation, mais celle de \cite{de_ketele_levaluation_1989} reste parmi les plus complètes.
\og Évaluer signifie
\begin{itemize} 
\item recueillir un ensemble d'informations suffisamment pertinentes, valides et fiables
\item et examiner le degré d'adéquation entre cet ensemble d'informations et un ensemble de critères adéquats aux objectifs fixés au départ ou ajustés en cours de route
\item en vue de prendre une décision \fg
\end{itemize}


Il explique que l'évaluation prépare à une décision mais n'en fait pas partie. 
Le fait de parler de décision est assimilé à un jugement, au fait d'apprécier une personne ou une action. 
Mais l'évaluation, contrairement au jugement, est un processus basé sur des critères explicites.

Pour cela dans son livre, ~\cite{roegiers_lecole_2010} propose trois évaluations des acquis des élèves~:
\begin{itemize}
\item \og en début d'année sur les performances des élèves pour décider si l'on peut commencer les apprentissages comme prévu \fg ~;
\item \og en cours d'année sur les performances d'un élève pour décider s'il est nécessaire de mettre en place une procédure de remédiation \fg ~;
\item \og en fin d'année sur les performances d'un élève pour décider s'il peut passer dans la classe supérieure \fg .
\end{itemize}

Ces trois types d'évaluations se déroule à deux échelles celle de l'élève et celle de la classe.
Concernant l'élève~:
\begin{itemize}
\item L'évaluation qui s'effectue avant l'apprentissage s'appelle l'évaluation d'orientation. Elle permet de diagnostiquer les forces et les faiblesses pour orienter l'élève vers un type d'apprentissage plus adapté. C'est pour cela que l'on parle également d'évaluation diagnostique.
\item L'évaluation qui s'effectue durant l'apprentissage permet de détecter les difficultés au niveau de chaque élève afin d'y remédier. Cette évaluation s'appelle évaluation formative.
\item L'évaluation qui s'effectue en fin d'apprentissage, fin d'année, permet de valider les acquis des élèves, c'est l'évaluation sommative aussi appelée évaluation certificative lorsqu'elle aboutit à la délivrance d'un diplôme ou l'accès à la formation du niveau supérieur.
\end{itemize}

\subsection{Un modèle d'évaluation}

Dans leur livre~\cite{doyon_faire_1991} présente un modèle d'évaluation formative qui fait participer l'élève du primaire à l'évaluation de ses apprentissages. 
Les auteurs proposent un modèle d'auto-évaluation qui a pour objectif d'amener l'élève à devenir de plus en plus autonome.
Ils décrivent un processus d'auto-évaluation cyclique composé de quatre phases qui se déroule à l'intérieur d'une séquence d'apprentissage.
%\begin{enumerate}
%\item Phase de planification~: \og elle consiste à déterminer avec les élèves les objectifs d'apprentissage poursuivis au cours de la séquence, à préciser les critère d'évaluation par des indicateurs d'observation \fg
%\item Phase de réalisation~: \og elle consiste à réaliser les activités d'évaluation prévues, à faire participer l'élève à l'évaluation de ses apprentissages et à consigner les résultats d'évaluation \fg
%\item Phase de communication des résultats~: \og elle consiste à préparer la communication des résultats aux parents, à organiser et à réaliser une rencontre parents-enfants au moment de la remise du bulletin \fg
%\item Phase de prise de décision~: \og  \fg
%\end{enumerate}

Voici la synthèse des quatre phases du processus d'auto-évaluation et les étapes qu'elles renferment~:
\begin{enumerate}
\item Phase de planification
	\begin{itemize}
	\item Détermination des objectifs d'apprentissage poursuivis aux cours de la séquence.
	\item Précision des critères d'évaluation des apprentissages.
	\item Prévision des situations d'apprentissage et d'évaluation.
	\item Préparation des outils de consignation des résultats d'évaluation.
	\end{itemize}
\item Phase de réalisation
	\begin{itemize}
	\item Réalisation des activités d'évaluation.
	\item Autoévaluation de l'élève et consignation de ses observations.
	\item Co-évaluation et consignation des observations de l'enseignant.
	\item Conservation des résultats d'évaluation.
	\end{itemize}
\item Phase de communication des résultats
	\begin{itemize}
	\item Préparation de la communication aux parents.
	\item Organisation de la rencontre parents-enfants.
	\item Réalisation de la rencontre parents-enfants.
	\end{itemize}
\item Phase de prise de décision
	\begin{itemize}
	\item Examen rétrospectif de la rencontre parents-enfants.
	\item Prise de conscience par l'élève de son cheminement et sélection d'objectifs personnels prioritaires.
	\item Sélection d'objectifs collectifs à être poursuivis par la classe.
	\item Vérification des outils de travail de l'élève.
	\end{itemize}
\end{enumerate}

Leur approche étant destinée aux élèves du primaire, je n'ai pas retenu tout le cheminement pour chaque séquence.
Je me suis concentrée sur les deux premières phases, celle de planification et celle de réalisation.


Je me suis inspirée de leur modèle d'évaluation que j'ai adapté au cycle du secondaire.
Retour par les élèves sur leurs apprentissages pour se situer~:
\begin{itemize}
\item J'ai réussi avec facilité et suis allé au-delà de la tâche demandée. 
\item J'ai tout juste réussi à réaliser la tâche demandée.
\item Je n'ai pas réussi à réaliser la tâche demandée, même avec de l'aide.
\end{itemize} 


%%% ----------------- Dans le collège

\section{Mise en place au sein de mon établissement}

Au sein de mon établissement nous utilisons le logiciel Sacoche pour saisir les évaluations et effectuer un suivi des compétences des élèves.
Ce logiciel regroupe les compétences du cycle, lors de la saisie d'une évaluation, l'enseignant choisi les compétences qu'il souhaite évaluer ainsi que le niveau atteint par chacun des élèves.

Les compétences sont évaluées suivant quatre niveaux d'acquisition~:
\begin{itemize}
\item Niveau 1~: Non acquis, ce niveau est identifié dans le collège par deux points rouges. L'élève n'a pas atteint l'objectif.
\item Niveau 2~: En voie d'acquisition, identifié par un point rouge. L'élève a eu quelques difficultés et n'a pas entièrement atteint l'objectif.
\item Niveau 3~: Acquis, identifié par un point vert. L'élève à atteint l'objectif.
\item Niveau 4~: Expert, identifié par deux points verts. L'élève à largement atteint l'objectif qui était fixé.
\end{itemize}

L'évaluation par compétence permet de mettre en évidence les différents savoirs à la différence d'une note globale qui n'identifie pas les points forts et difficultés de l'élève.

%%% -----------------
%%% -----------------

\section{Les compétences}

On identifie quelqu'un de compétent comme ayant des acquis (connaissances, savoir-faire, procédures, etc.) et sachant les mobiliser pour résoudre un problème donné.
En d'autres mots, être compétent c'est vouloir, pouvoir et savoir.

Les savoirs regroupent les connaissances théoriques, la connaissance d'un vocabulaire, des lois et normes.

Le savoir-faire est la maîtrise des modes opératoires et des processus dans une situation spécifique.

Le savoir-être est la façon de se comporter.

Afin de prouver sa compétence, les élèves mobilisent les savoirs, savoir-faire et savoir-être

\subsection{Un outil pour évaluer les compétences}

Le fait d'évaluer les élèves par compétences permet de connaître les notions travaillées ainsi que les attentes de l'institution.
Le logiciel \textit{SACoche} pour Suivi d'Acquisition de Compétences permet de saisir les compétences et ainsi de d'identifier à tout moment les compétences acquises ou non par les élèves.
De plus SACoche permet d'associer des liens à des items ce qui permet aux élèves de retravailler les compétences non acquises et donc de faire une remédiation.


\subsection{Une Évaluation Par Contrat de Confiance (EPCC)}

Dans la littérature !,??  (EPCC).





\subsection{Zone Proximale de Développement (ZPD)}

D'après V??? la Zone Proximale de Développement  est \og La distance entre le niveau de développement actuel (la résolution indépendante de problèmes), et le niveau de développement potentiel (la résolution de problèmes supervisée par un adulte ou en collaboration avec des pairs plus habiles) \fg.
 

------------------------------
Zone proximale de développement (ZPD) 


Zone proximale de développement
 Définition  
 \og La distance entre le niveau de développement actuel (la résolution indépendante de problèmes), et le niveau de développement potentiel (la résolution de problèmes supervisée par un adulte ou en collaboration avec des pairs plus habiles) \fg
  (Vygotsky, 1978; Bodrova \& Leong, 2011, p.62).

 \og Selon Vygotsky, ce que l'enfant peut accomplir aujourd'hui avec de l'aide, il sera en mesure de le faire demain de façon indépendante \fg{} 
 (Vygotsky, 1987; Bodrova \& Leong, 2011, p.62).
 Performance indépendante : indicateur important du développement, mais il n'est pas suffisant pour l'expliquer à lui seul ...               (Bodrova \& Leong, 2011)


------------------------------


