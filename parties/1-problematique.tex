
% Présentation du collège

J'enseigne dans le collège Henry de Montherlant situé à Neuilly-en-Thelle.
L'établissement est situé dans une zone rurale, la majorité des élèves viennent des communes voisines par des bus scolaires.
Ce collège accueille à la rentrée 2017-2018 822 élèves inscrits dont 770 demi-pensionnaires et 52 externes.
Chaque niveau est découpé en 7 classes dont une classe de SEGPA.
L'établissement regroupe 74 personnels dont~:
\begin{itemize}
\item 6 Personnels administratifs, techniques, santé et de service
\item 12 Personnels de vie scolaire
\item 54 Personnels enseignants
\item 2 Rattachés administratifs
\end{itemize}

Ce collège est un collège sans note, je me suis demandée comment évaluer les élèves moi qui étais habitué aux notes.

% Situation déclanchante
Lors de l'évaluation sommative sur la première séquence j'ai constaté une faible réussite.
J'ai donc remis en cause tout d'abord mon enseignement puis me suis ensuite interrogée sur la façon d'évaluer les élèves.

% Problématique
J'en suis donc arrivé à la problématique suivante : 

Comment évaluer les élèves ?

Quels sont les outils qui permettent une évaluation par compétence?

Comment peut-on rendre les élèves acteurs de leur évaluation ?

