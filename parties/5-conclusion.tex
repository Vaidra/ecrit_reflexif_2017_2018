
J'ai débuté cette année scolaire 2017-2018 avec une certaine appréhension sur l'évaluation par compétences.
Notamment lorsque j'ai découvert que l'établissement dans lequel j'enseigne est un collège sans note.
Après cette réflexion sur l'évaluation et les compétences j'estime que cela permet de cibler davantage les points forts et les points faibles des élèves.

Concernant l'évaluation j'ai eu l'occasion de mettre en place différents systèmes.
J'ai ressenti qu'en évaluant les élèves durant les activités ils sont plus investis.

Le fait de leur donner des objectifs à atteindre pour valider les compétences semble les motiver, ils se réfèrent régulièrement à la fiche d'évaluation afin de faire appel au professeur pour qu'il valide leur niveau d'acquisition.


Par la suite je vais continuer ma démarche sur l'évaluation par compétences, notamment en m'inspirant de l'approche d'Antibi avec le système d'évaluation par contrat de confiance qui me semble intéressante autant pour les élèves que pour l'enseignant. 

Il serait tout aussi intéressant de s'interroger sur la taxonomie qui peut-être mise en place tout au long du cycle 4.
Mais aussi sur le fait de juger qu'un élève est compétent.
Si l'on décompose une compétence en sous compétences, à partir de quel moment un élève est-il considérer comme compétent~?
Peu d'élève arrive à avoir un niveau expert à chaque fois, le fait d'imposer un seuil de réussite permettrai aux élèves d'être considérer compétent sans pour autant maitriser la compétence complètement.

