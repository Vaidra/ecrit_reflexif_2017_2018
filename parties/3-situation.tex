
%%% ------------------------

\section{Évaluation durant les travaux pratiques}

Durant la séquence 3 intitulée \og Comment aménager un lotissement~?\fg j'ai commencé par présenter les compétences travaillées par les élèves durant la première séance.
Puis j'ai expliquer l'activité sur laquelle ils seront évaluer~: afin de faire venir de nouveaux habitants dans sa commune, le maire de Neuilly-en-Thelle souhaite créer un nouveau lotissement pour accueillir des terrains à bâtir.
L'objectif est d'installer un maximum de terrains à bâtir dans l'espace disponible mais afin de garantir le confort des nouveaux habitants, le constructeur préconise les dimensions à respecter.
Bien entendu chaque terrain à bâtir doit être relié à la route.

Pour cela les élèves sont repartis par groupe de 4 et ont à leur dispositions un plan du lotissement à l'échelle $1:50$, des rectangles, chacun représentant un terrain à bâtir avec les dimensions imposées ainsi que des feuilles dans lesquelles ils devront découper les routes, la largeur des routes est aussi imposée.

Les élèves travaillent sur le cahier des charges ainsi que la représentation d'une solution technique à un problème posé.
Ils sont évalués sur leur réalisation, c'est-à-dire le respect du cahier des charges ainsi que l'optimisation de l'espace occupé.

Durant l'activité les élèves étaient investis dans leur travail, ils ont testés plusieurs répartitions spatiales afin de trouver la plus optimisée.
Les compétences que les élèves ont travaillées sont majoritairement acquises excepté un groupe qui n'a positionné aucune route pour relier les terrains à bâtir au reste de la commune.


%%% ------------------------

\section{Donner les clés de la réussite aux élèves}

Lors du lancement de la séquence 5 intitulée \og Comment évoluent les objets techniques~?\fg j'ai présenté les compétences travaillées mais j'ai aussi fournis au élèves une fiche qui détaille les objectifs à atteindre pour les différents niveaux d'acquisition. 
Chaque compétence est découpée en 4 objectifs décrit sur la feuille distribuée, chaque objectif validé rapporte 1 point. Suivant le nombre de point sur chaque compétence l'élève obtient soit Insuffisant (0-1 point), Fragile (2 points), Acquis (3 points), Expert (4 Points).

L'objectif de l'activité évaluée était de réaliser, par binôme, une frise chronologique sur l'évolution d'un objet technique avec le logiciel OpenOffice Dessin.
Lors de l'activité  cela  
Présentation du mode d'évaluation.

\section{Permettre au élèves de choisir leur évaluation}

A la fin de la séquence 6 intitulée \og Comment fonctionnent les objets techniques~?\fg
Évaluation différenciée sur la chaîne d'énergie.

%%% ------------------------

\section{Permettre à chacun de se situer }
Séquence 7~: Comment fonctionne une écluse~?
Mise en place d'une évaluation différencier.

%%% ------------------------


