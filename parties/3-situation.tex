
%%% ------------------------

\section{Évaluation durant les travaux pratiques}

Durant la séquence 2 intitulée \og Comment aménager un lotissement~? \fg \ j'ai commencé par présenter les compétences travaillées par les élèves durant la première séance.
Puis j'ai expliqué l'activité sur laquelle ils seront évalués~: afin de faire venir de nouveaux habitants dans sa commune, le maire de Neuilly-en-Thelle souhaite créer un nouveau lotissement pour accueillir des terrains à bâtir.
L'objectif est d'installer un maximum de terrains à bâtir dans l'espace disponible mais afin de garantir le confort des nouveaux habitants, le constructeur préconise des dimensions à respecter.
Bien entendu chaque terrain à bâtir doit être relié à la route.

Pour cela les élèves sont répartis par groupe de 4 et ont à leur disposition un plan du lotissement à l'échelle $1:50$, des rectangles, chacun représentant un terrain à bâtir avec les dimensions imposées ainsi que des feuilles dans lesquelles ils devront découper les routes, la largeur des routes est aussi imposée.

Les élèves travaillent sur le cahier des charges ainsi que la représentation d'une solution technique à un problème posé.
Ils sont évalués sur leur réalisation, c'est-à-dire le respect du cahier des charges ainsi que l'optimisation de l'espace occupé.

Durant l'activité les élèves étaient investis dans leur travail, ils ont testé plusieurs répartitions spatiales afin de trouver la plus optimisée.
Les compétences que les élèves ont travaillées sont majoritairement acquises excepté un groupe qui n'a positionné aucune route pour relier les terrains à bâtir au reste de la commune.


%%% ------------------------

\section{Donner les clés de la réussite aux élèves}

Lors du lancement de la séquence 4 intitulée \og Comment évoluent les objets techniques~?\fg j'ai présenté les compétences travaillées mais j'ai aussi fourni aux élèves une fiche qui détaille les objectifs à atteindre pour les différents niveaux d'acquisition. 
Chaque compétence est découpée en 4 objectifs décrit sur la feuille distribuée, chaque objectif validé rapporte 1 point. Suivant le nombre de point sur chaque compétence l'élève obtient soit Insuffisant (0-1 point), Fragile (2 points), Acquis (3 points), Expert (4 Points).

L'objectif de l'activité évaluée était de réaliser, par binôme, une frise chronologique sur l'évolution d'un objet technique avec le logiciel OpenOffice Dessin.
Lors de l'activité les élèves  ??

Présentation du mode d'évaluation. ??

\section{Permettre aux élèves de choisir leur évaluation}

A la fin de la séquence 5 intitulée \og Comment fonctionnent les objets techniques~?\fg j'ai mis en place une évaluation différenciée.

Durant cette séquence les élèves ont étudiés la chaîne d'énergie de différents objets techniques ainsi que les blocs fonctionnels qui la compose.

Afin de vérifier l'acquisition de la compétence j'ai programmé une évaluation sur feuille.
Étant donné que durant les travaux dirigés certains élèves étaient en difficulté j'ai décider de mettre en place une évaluation différenciée, c'est-à-dire que j'ai proposé aux élèves deux sujets.

Le premier sujet parcours toutes les connaissances vues en classe et demande tout d'abord de représenter le schéma de la chaîne d'énergie sans aucune indication puis de réaliser la chaîne d'énergie de différents objets techniques.
Ce sujet (voir Annexe~\ref{annexe:evaluation_chaine_energie_A}) évalue la compétence sur tout le niveau d'acquisition (Insuffisant, Fragile, Acquis, Expert).

Le second (voir Annexe~\ref{annexe:evaluation_chaine_energie_B}) sujet s'adresse aux élèves ayant eu des difficulté durant les travaux dirigés.
Il évalue la compétence sur les trois premier niveaux d'acquisition, cela signifie que les élève qui ont choisi ce sujet ne pourrons pas valider la compétence avec le niveau expert.
En effet, ce sujet demande de compléter le schéma de la chaîne d'énergie qui est représenté puis de réaliser ce schéma avec plusieurs objets techniques qui ont étés vu en cours.



J'ai remarqué que certains élèves ne savent pas se positionner sur leur connaissances et compétences.




%%% ------------------------

\section{Permettre à chacun de se situer }

Séquence 7~: Comment fonctionne une écluse~?
Mise en place d'une évaluation différenciée.

%%% ------------------------


