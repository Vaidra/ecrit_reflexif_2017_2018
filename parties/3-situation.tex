
%%% ------------------------

\section{Évaluation durant les travaux pratiques}

Durant la séquence 2 intitulée \og Comment aménager un lotissement~? \fg \ j'ai commencé par présenter les compétences travaillées par les élèves durant la première séance.
Puis j'ai expliqué l'activité sur laquelle ils seront évalués~: afin de faire venir de nouveaux habitants dans sa commune, le maire de Neuilly-en-Thelle souhaite créer un nouveau lotissement pour accueillir des terrains à bâtir.
L'objectif est d'installer un maximum de terrains à bâtir dans l'espace disponible mais afin de garantir le confort des nouveaux habitants, le constructeur préconise des dimensions à respecter.
Bien entendu chaque terrain à bâtir doit être relié à la route.

Pour cela les élèves sont répartis par groupe de 4 et ont à leur disposition un plan du lotissement à l'échelle $1:50$, des rectangles, chacun représentant un terrain à bâtir avec les dimensions imposées ainsi que des feuilles dans lesquelles ils devront découper les routes, la largeur des routes est aussi imposée.

Les élèves travaillent sur le cahier des charges ainsi que la représentation d'une solution technique à un problème posé.
Ils sont évalués sur leur réalisation, c'est-à-dire le respect du cahier des charges ainsi que l'optimisation de l'espace occupé.

Durant l'activité les élèves étaient investis dans leur travail, ils ont testé plusieurs répartitions spatiales afin de trouver la plus optimisée.
Les compétences que les élèves ont travaillées sont majoritairement acquises excepté un groupe qui n'a positionné aucune route pour relier les terrains à bâtir au reste de la commune.


%%% ------------------------

\section{Donner les clés de la réussite aux élèves}

Lors du lancement de la séquence 4 intitulée \og Comment évoluent les objets techniques~?\fg j'ai présenté les compétences travaillées mais j'ai aussi fourni aux élèves une fiche qui détaille les objectifs à atteindre pour les différents niveaux d'acquisition. 
Chaque compétence est découpée en trois objectifs décrit sur la feuille distribuée, chaque objectif validé rapporte 1 point. Suivant le nombre de point sur chaque compétence l'élève obtient soit Insuffisant (0 point), Fragile (1 points), Acquis (2 points), Expert (3 Points).

L'objectif de l'activité évaluée était de réaliser, par binôme, une frise chronologique sur l'évolution d'un objet technique de leur choix avec le logiciel \textit{OpenOffice Dessin}.

Lors de l'activité les élèves étaient impliqués dans leur évaluation, ils ont régulièrement regarder la fiche d'évaluation pour se positionner.


\section{Permettre aux élèves de choisir leur évaluation}

A la fin de la séquence 5 intitulée \og Comment fonctionnent les objets techniques~?\fg j'ai mis en place une évaluation différenciée.

Durant cette séquence les élèves ont étudiés la chaîne d'énergie de différents objets techniques ainsi que les blocs fonctionnels qui la compose.

Afin de vérifier l'acquisition de la compétence j'ai programmé une évaluation sur feuille.
Étant donné que durant les travaux dirigés certains élèves étaient en difficulté j'ai décider de mettre en place une évaluation différenciée, c'est-à-dire que j'ai proposé aux élèves deux sujets.

Le premier sujet parcours toutes les connaissances vues en classe et demande tout d'abord de représenter le schéma de la chaîne d'énergie sans aucune indication puis de réaliser la chaîne d'énergie de différents objets techniques.
Ce sujet (voir Annexe~\ref{annexe:evaluation_chaine_energie_A}) évalue la compétence sur tout le niveau d'acquisition (Insuffisant, Fragile, Acquis, Expert).

Le second (voir Annexe~\ref{annexe:evaluation_chaine_energie_B}) sujet s'adresse aux élèves ayant eu des difficulté durant les travaux dirigés.
Il évalue la compétence sur les trois premiers niveaux d'acquisition, cela signifie que les élèves qui ont choisi ce sujet ne pourront pas valider la compétence avec le niveau expert.
En effet, ce sujet demande de compléter le schéma de la chaîne d'énergie qui est représenté puis de réaliser ce schéma avec plusieurs objets techniques qui ont étés vu en cours.

J'ai constaté que certains élèves ne savent pas se positionner sur leur connaissances et compétences et donc hésite sur le sujet qu'ils souhaitent traiter.


%%% ------------------------

\section{Permettre à chacun de se situer }

La séquence 7 est intitulée \og Comment fonctionne une écluse~? \fg{}.
Cette séquence se déroule en deux parties~:
\begin{itemize}
\item Découverte des éléments de l'ouvrage d'art puis du fonctionnement de celui-ci en utilisant une application flash. Les élèves devront réaliser une présentation en utilisant \textit{OpenOffice Présentation} dans laquelle ils détailleront les étapes pour permettre à l'écluse de laisser passer un bateau.
\item Étude des étapes de fonctionnement, après une explication sur la réalisation d'un organigramme, les élèves devront réaliser l'organigramme de l'écluse puis de divers objets techniques.
\end{itemize} 

Durant la présentation de la première partie sur le fonctionnement d'une l'écluse, j'ai expliqué aux élèves qu'ils seront évalués sur la réalisation d'une présentation sur le fonctionnement d'une l'écluse.
Pour cela, je leur ai présenté puis distribué la fiche d'évaluation sur laquelle je détaille les objectifs à atteindre pour valider la compétence (voir Annexe~\ref{annexe:evaluation_fct_ecluse}).
En plus de cette évaluation, je demande un retour des élèves sur leur ressentit durant l'activité.

La seconde activité porte sur la réalisation d'organigrammes.
Les élèves ont tout d'abord complété des organigrammes à trous puis ils ont réaliser les organigrammes uniquement à l'aide d'un texte qui décrit le fonctionnement de l'objet technique.

Afin d'aider les élèves à se positionner, j'ai donc choisi d'utiliser les outils numérique et notamment le logiciel Tactileo. 
Ce logiciel m'a permis de créer un contenu multimédia interactif.
Les élèves s'enregistrent avec leur nom et prénom puis visualisent et répondent aux questions du contenu.
Il peut être utilisé via un ordinateur, une tablette ou un smartphone.
Pour cette séquence j'ai personnaliser un parcours, c'est-à-dire que, si un élève n'a pas répondu correctement à une question, je le réoriente vers une ressource numérique pour qu'il revoit la notion puis vers une question similaire. Cet outil permet de faire une personnalisation et différentiation en créant des parcours complets.

La réalisation de ce parcours par les élèves leur à permis de pouvoir se positionner pour le choix d'un des deux sujets d'évaluation.


\section{Associer les compétences aux questions et objectifs}

Je n'ai pas eu l'occasion de pouvoir poursuivre davantage ma réflexion sur l'évaluation des compétences mais j'ai réfléchi à quelques améliorations que je peux apporter.

Pour la séquence suivante je penses intégrer directement l'évaluation dans le sujet de travaux pratiques que je proposerai aux élèves.
Au fur et à mesure de la lecture du sujet, pour chaque objectif les élèves verront directement les sous-objectifs à atteindre pour valider le niveau de compétence associé.

Concernant les sujets d'évaluation sur feuille, je préciserai pour chaque question la compétence évaluée.
Je proposerai également de donner comme devoir à la maison, après les résultats d'évaluation, les sujets plus compliqués aux élèves ayant fait le sujet plus faibles et ayant bien réussi.
Cela permettra à ces élèves de se rendre compte de la différence de difficulté avec leur sujet qu'ils ont choisi et peut-être les inciter à la prochaine évaluation de choisir le sujet permettant de valider la maitrise experte de la compétence.

De plus la réflexion de \cite{antibi2007notes} concernant le système d'évaluation par contrat de confiance me semble intéressant et j'envisage de le mettre en place pour les séquences suivantes.

%%% ------------------------


