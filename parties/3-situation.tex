
%%% ------------------------

\section{Évaluation durant les travaux pratiques}
Durant la séquence intitulée \og Comment aménager un lotissement~?\fg j'ai commencé par présenter les compétences travaillées par les élèves durant la première séance. Puis j'ai expliquer l'activité sur laquelle ils seront évaluer~: afin de faire venir de nouveaux habitants dans sa commune, le maire de Neuilly-en-Thelle souhaite créer un nouveau lotissement pour accueillir des terrains à bâtir.
L'objectif est d'installer un maximum de terrains à bâtir dans l'espace disponible mais afin de garantir le confort des nouveaux habitants, le constructeur préconise les dimensions à respecter.

Pour cela les élèves sont repartis par groupe de 4 et ont à leur dispositions un plan du lotissement à l'échelle $1:50$, des rectangles, chacun représentant un terrain à bâtir avec les dimensions imposées ainsi que des feuilles dans lesquelles ils devront découper les routes, la largeur des routes est aussi imposée.




Évaluation de la réalisation des travaux pratiques durant leur réalisation.

%%% ------------------------

\section{Donner les clés de la réussite aux élèves }
Séquence 5~: Comment évoluent les objets techniques~?
Présentation du mode d'évaluation.

Évaluation différenciée.

%%% ------------------------

\section{Permettre à chacun de se situer }
Séquence 6~: Comment fonctionne une écluse~?
Mise en place d'une évaluation différencier.

%%% ------------------------

Dans leur livre~\cite{doyon_faire_1991} présente un modèle d'évaluation formative qui fait participer l'élève du primaire à l'évaluation de ses apprentissages. 
Les auteurs propose un modèle d'auto-évaluation qui a pour objectif d'amener l'élève à devenir de plus en plus autonome.
Ils décrivent un processus d'auto-évaluation cyclique composé de quatre phases qui se déroule à l'intérieur d'une séquence d'apprentissage.
%\begin{enumerate}
%\item Phase de planification~: \og elle consiste à déterminer avec les élèves les objectifs d'apprentissage poursuivis au cours de la séquence, à préciser les critère d'évaluation par des indicateurs d'observation \fg
%\item Phase de réalisation~: \og elle consiste à réaliser les activités d'évaluation prévues, à faire participer l'élève à l'évaluation de ses apprentissages et à consigner les résultats d'évaluation \fg
%\item Phase de communication des résultats~: \og elle consiste à préparer la communication des résultats aux parents, à organiser et à réaliser une rencontre parents-enfants au moment de la remise du bulletin \fg
%\item Phase de prise de décision~: \og  \fg
%\end{enumerate}

Voici la synthèse des quatre phases du processus d'auto-évaluation et les étapes qu'elles renferment~:
\begin{enumerate}
\item Phase de planification
	\begin{enumerate}
	\item Détermination des objectifs d'apprentissage poursuivis aux cours de la séquence.
	\item Précision des critères d'évaluation des apprentissages.
	\item Prévision des situations d'apprentissage et d'évaluation.
	\item Préparation des outils de consignation des résultats d'évaluation.
	\end{enumerate}
\item Phase de réalisation
	\begin{enumerate}
	\item Réalisation des activités d'évaluation.
	\item Autoévaluation de l'élève et consignation de ses observations.
	\item Co-évaluation et consignation des observations de l'enseignant.
	\item Conservation des résultats d'évaluation.
	\end{enumerate}
\item Phase de communication des résultats
	\begin{enumerate}
	\item Préparation de la communication aux parents.
	\item Organisation de la rencontre parents-enfants.
	\item Réalisation de la rencontre parents-enfants.
	\end{enumerate}
\item Phase de prise de décision
	\begin{enumerate}
	\item Examen rétrospectif de la rencontre parents-enfants.
	\item Prise de conscience par l'élève de son cheminement et sélection d'objectifs personnels prioritaires.
	\item Sélection d'objectifs collectifs à être poursuivis par la classe.
	\item Vérification des outils de travail de l'élève.
	\end{enumerate}
\end{enumerate}

Leur approche étant destinée aux élèves du primaire, je n'ai pas retenu tout le cheminement pour chaque séquence.
Je me suis concentré sur les deux premières phases, celle de planification et celle de réalisation.


Je me suis inspiré de leur modèle approche d'évaluation que j'ai adapter au cycle du secondaire.
Retour par les élèves sur leur apprentissages pour se situer~:
\begin{itemize}
\item J'ai réussi avec facilité et suis allé au-delà de la tâche demandée. 
\item J'ai tout juste réussi à réaliser la tâche demandée.
\item Je n'ai pas réussi à réaliser la tâche demandée, même avec de l'aide.
\end{itemize} 

