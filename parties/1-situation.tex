% Mise en place de l'évaluation

\section{Évaluation des connaissances}

Ma première séquence était intitulée \og Qu'est-ce qu'un cahier des charges fonctionnel~?\fg. 
Pour cette première séquence je n'ai pas réalisé d'évaluation diagnostique afin de déterminer les connaissances des élèves sur le cahier des charges.
Cela fut une première erreur de ma part car je pensais effectuer des rappels avant d'aller plus loin, or pour les élèves les notions de \og diagramme bête à cornes\fg ~et de \og diagramme pieuvre\fg ~n'évoquaient rien. J'ai donc passé une séance complète sur la réalisation de ces deux diagrammes qui étaient inconnus aux élèves.
Si j'avais effectué une rapide évaluation diagnostique cela m'aurait fait gagner du temps.

Concernant la structure de la séquence sur la réalisation du cahier des charges je demandais énormément d'attention de la part des élèves.
Effectivement mon cours s'est déroulé de façon magistral avec une partie travaux dirigés durant laquelle les élèves ont réalisés les deux types de diagrammes.
Les élèves n'ont pas communiqués durant les exercices et s'en ai suivie une phase de correction au tableau par des élèves volontaires.

Afin de vérifier que les élèves puissent faire par eux même un diagramme bête à cornes je leur ai demandé de réaliser sur une feuille ce diagramme sur différents objets de leur quotidien.

J'ai ensuite poursuivi le cours sur le cahier des charges notamment sur les fonctions principales et fonctions contraintes.

A la fin de la séquence j'ai prévenu les élèves qu'une évaluation aurait lieu la semaine suivante. Cette évaluation porterait sur la réalisation de deux diagrammes et l'identification de la fonction principale et des fonctions contraintes.

L'évaluation que j'ai réalisé se trouve Annexe~\ref{annexe:evaluation_cdcf}.

\section{Critique}

Cette évaluation est une évaluation dite \og classique \fg, elle consiste à évaluer les connaissances des élèves.

Si j'avais obtenu beaucoup de bons résultats je pense que je me serai interrogé sur la qualité de mon évaluation.
Le fait ici d'avoir de mauvaises notes m'a "rassuré" car je n'avais pas sous estimé la capacité des élèves.

Après la lecture de livre d'Antibi sur \og constante macabre \fg je me suis rendu compte qu'il y a de grandes variations de notation suivant les examinateurs, mais aussi d'un même examinateur dans le temps.
De plus l'existance de la \og constante macabre \fg~\cite{antibi2003constante} est aussi un facteur d'influence.
Ce phénomène consiste à attribuer, quelle que soit la qualité des copies, un pourcentage de mauvaises notes.

En lisant ce livre je me suis identifiée aux enseignants qui cherchaient à obtenir une répartition des notes \og 1/3, 1/3, 1/3\fg.
Cette lecture m'a permis de prendre conscience de ce phénomène et de m'interroger sur la façon dont je pouvais évaluer les compétences des élèves en laissant de côté cette constante macabre.

Pour supprimer \og constante macabre \fg, Antibi propose de mettre en place une évaluation par objectifs.
Cela consiste à déterminer des objectifs clairs et précis que l'élève doit atteindre afin de réussir un contrôle.

Je me suis donc inspiré de cela pour évaluer les compétences des élèves non pas avec un contrôle mais avec un objectif à atteindre.
