% Mise en place de l'évaluation

\section{Évaluation des connaissances}

Ma première séquence était intitulée \og Qu'est-ce qu'un cahier des charges fonctionnelles~?\fg. 
Pour cette première séquence je n'ai pas réalisé d'évaluation diagnostique afin de déterminer les connaissances des élèves sur le cahier des charges.
Cela fut une première erreur de ma part car je pensais effectuer des rappels avant d'aller plus loin, or pour les élèves les notions de \og diagramme bête à cornes\fg et de \og diagramme pieuvre\fg n'évoquait rien. J'ai donc passé une séance complète sur la réalisation de ces deux diagrammes qui étaient inconnus aux élèves.
Si j'avais effectué une rapide évaluation diagnostique cela m'aurait fait gagner du temps.

Concernant la structure de la séquence que j'ai réalisé sur la réalisation du cahier des charges je demandais énormément d'attention de la part des élèves.
Effectivement mon cours s'est déroulé de façon magistral avec une partie travaux dirigés durant laquelle les élèves ont réalisés les deux types de diagrammes.
Les élèves n'ont pas communiqués durant les exercices et s'en ai suivie une phase de correction au tableau par des élèves volontaire.

Afin de vérifier que les élèves puissent faire par eux même un diagramme bête à cornes je leur ai demandé de réaliser sur une feuille ce diagramme sur différents objets de leur quotidien.

J'ai ensuite poursuivit le cours sur le cahier des charges notamment sur les fonctions principales et fonctions contraintes.

A la fin de la séquence j'ai prévenu les élèves qu'une évaluation aurait lieu la semaine suivante. Cette évaluation porterait sur la réalisation de deux diagrammes et l'identification de la fonction principale et des fonctions contraintes.

L'évaluation que j'ai réalisé se trouve Annexe~\ref{annexe:evaluation_cdcf}.

\section{Critique}

Cette évaluation est une évaluation dite \og classique \fg qui consiste à 

