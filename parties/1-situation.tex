% Mise en place de l'évaluation

\section{Évaluation des connaissances}

Ma première séquence était intitulée \og Qu'est-ce qu'un cahier des charges fonctionnel~?\fg. 
Pour cette première séquence je n'ai pas réalisé d'évaluation diagnostique afin de déterminer les connaissances des élèves sur le cahier des charges.
Cela fut une première erreur de ma part car je pensais effectuer des rappels avant d'aller plus loin, or pour les élèves les notions de \og diagramme bête à cornes\fg ~et de \og diagramme pieuvre\fg ~n'évoquaient rien. J'ai donc passé une séance complète sur la réalisation de ces deux diagrammes qui étaient inconnus aux élèves.
Si j'avais effectué une rapide évaluation diagnostique cela m'aurait fait gagner du temps.

Concernant la structure de la séquence sur la réalisation du cahier des charges je demandais énormément d'attention de la part des élèves.
Effectivement mon cours s'est déroulé de façon magistral avec une partie travaux dirigés durant laquelle les élèves ont réalisés les deux types de diagrammes.
Les élèves n'ont pas communiqués durant les exercices et s'en ai suivie une phase de correction au tableau par des élèves volontaires.

Afin de vérifier que les élèves puissent faire par eux même un diagramme bête à cornes je leur ai demandé de réaliser sur une feuille ce diagramme sur différents objets de leur quotidien.

J'ai ensuite poursuivi le cours sur le cahier des charges notamment sur les fonctions principales et fonctions contraintes.

A la fin de la séquence j'ai prévenu les élèves qu'une évaluation aurait lieu la semaine suivante. Cette évaluation porterait sur la réalisation de deux diagrammes et l'identification de la fonction principale et des fonctions contraintes.

L'évaluation que j'ai réalisé se trouve Annexe~\ref{annexe:evaluation_cdcf}.

\section{Critique}

Cette évaluation est une évaluation dite \og classique \fg, elle consiste à évaluer les connaissances des élèves.

Si j'avais obtenu beaucoup de bons résultats je pense que je me serai interrogé sur la qualité de mon évaluation.
Le fait ici d'avoir de mauvaises notes m'a "rassuré" car je n'avais pas sous estimé la capacité des élèves.

Après la lecture de livre d'Antibi sur \og constante macabre \fg je me suis rendu compte qu'il y a de grandes variations de notation suivant les examinateurs, mais aussi d'un même examinateur dans le temps.
De plus l'existance de la \og constante macabre \fg~\cite{antibi2003constante} est aussi un facteur d'influence.
Ce phénomène consiste à attribuer, quelle que soit la qualité des copies, un pourcentage de mauvaises notes.

En lisant ce livre je me suis identifiée aux enseignants qui cherchaient à obtenir une répartition des notes \og 1/3, 1/3, 1/3\fg.
Cette lecture m'a permis de prendre conscience de ce phénomène et de m'interroger sur la façon dont je pouvais évaluer les compétences des élèves en laissant de côté cette constante macabre.

Pour supprimer \og constante macabre \fg, s~\cite{antibi2007notes} propose de mettre en place une évaluation par objectifs.
Cela consiste à déterminer des objectifs clairs et précis que l'élève doit atteindre afin de réussir un contrôle.

Je me suis donc inspiré de cela pour évaluer les compétences des élèves non pas avec un contrôle mais avec un objectif à atteindre.

------------------------------
??
d’Evaluation Par Contrat de Confiance (EPCC).


 Sommaire du dossier
Pourquoi le logiciel s'appelle-t-il SACoche ?
SACoche parce que "SAC" signifie "Suivi d'Acquisition de Compétences".
SACoche avec "oche" pour compléter l'acronyme en référence à d'autres projets de Sésamath : Mathenpoche, Tracenpoche, Instrumenpoche, Casenpoche.
SACoche parce qu'un professeur qui évalue par compétences, "ça coche" des compétences.
SACoche pour la sacoche du professeur...
Merci à Sylvain pour cette riche idée, et merci à Arnaud pour le logo élaboré en un rien de temps ! 

 Sommaire du dossier
Pourquoi évaluer par compétences ?
Parce que l'évaluation par compétences permet à chacun de savoir à tout moment ce que l'élève a acquis et ce qu'il maitrise moins ou pas du tout. Si on associe aux items un lien vers des ressources, l'élève a alors la possibilité de retravailler ses points faibles, voir de réviser en amont. Un "10/20" sur une copie n'apporte pas cet éclairage et ne permet pas cette remédiation.
Parce que l'évaluation par compétences permet une reconnaissance de l'école en indiquant à l'élève, même en difficulté, ce qu'il sait faire : c'est important pour la confiance en soi, l'estime de soi, et le respect de soi. Un "3/20" sur une copie ne renvoie pas cette image.
Parce que l'évaluation par compétences permet d'estimer les acquisitions sur la durée, en favorisant les évaluations récentes : elle donne à l'élève le droit à l'erreur, elle valorise une remédiation réussie, tout en remettant aussi en question les acquis (rien n'est figé !). C'est différent d'un principe de validation, tel que pratiqué avec Gibii pour le B2i.
Parce que l'évaluation par compétences permet à l'élève et sa famille d'avoir explicitement connaissance des notions travaillées, de savoir quelles sont les attentes de l'école.


Zone proximale de développement (ZPD) 

------------------------------
Zone proximale de développement (ZPD) 


Zone proximale de développement
 Définition  
 \og La distance entre le niveau de développement actuel (la résolution indépendante de problèmes), et le niveau de développement potentiel (la résolution de problèmes supervisée par un adulte ou en collaboration avec des pairs plus habiles) \fg
  (Vygotsky, 1978; Bodrova \& Leong, 2011, p.62).

 \og Selon Vygotsky, ce que l'enfant peut accomplir aujourd'hui avec de l'aide, il sera en mesure de le faire demain de façon indépendante \fg{} 
 (Vygotsky, 1987; Bodrova \& Leong, 2011, p.62).
 Performance indépendante : indicateur important du développement, mais il n'est pas suffisant pour l'expliquer à lui seul ...               (Bodrova \& Leong, 2011)


------------------------------


 