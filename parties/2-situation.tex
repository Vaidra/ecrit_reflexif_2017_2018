
Il existe différents types d'évaluation~:
\begin{itemize}
\item L'évaluation diagnostique~: elle permet de déterminer le niveaux des élèves avant l'apprentissage.
\item L'évaluation formative~: elle permet de mettre en évidence les progrès des élèves pendant l'apprentissage. Cette évaluation permet de pouvoir mettre en place une différenciation afin d'accompagner au mieux les élèves.
\item L'évaluation sommative~: elle permet de faire un bilan sur les acquis des élèves après l'apprentissage.
\end{itemize}


Il y a quatre niveaux d'acquisition sur les compétences~:
\begin{itemize}
\item Niveau 1~: Non acquis, ce niveau est identifié dans le collège par deux points rouges.
\item Niveau 2~: En voie d'acquisition, identifié par un point rouge.
\item Niveau 3~: Acquis, ?? L'élève à atteint l'objectif
\item Niveau 4~: Expert, ?? L'élève à largement atteint l'objectif qui était fixé.
\end{itemize}


L'évaluation par compétence permet de mettre en évidence les différents savoirs à la différence d'une note globale qui n'identifie pas les points forts et difficultés de l'élève.
